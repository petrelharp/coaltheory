

%%%%%%% %%%%%%%%
\section{Mate choice: pedigrees}

Now that we have some idea of what our data can tell us about the structure of the ancestral recombination graph,
we will describe some simple population models --
i.e.\ stochastic models of pedigrees.
We will then compute expected values of the statistics under some of these models.


\subsection{The diploid Moran model:}
Continuous-time, fixed population size, overlapping generations.
In continuous time, each individual, at rate 1, chooses another uniformly at random
with whom to mate;
they produce one offspring
that replaces a randomly chosen individual (possibly including the parents).

\subsection{The diploid Cannings model:}
Discrete time, varying population size, nonoverlapping generations.
Suppose at time $t$ there are $N_t$ members of the population,
which we take as a given trajectory.
Each pair of individuals could potentially produce some offspring;
let $X_{ij}$ denote the number produced in this event by pair $(i,j)$ with $i$ as the mother and $j$ as the father for each $1 \le i,j \le N_{t}$.
Suppose that the $X_{ij}$ are \emph{exchangeable},
i.e.\ that $(X_{ij})_{i,j=1}^{N_{t}} \deq (X_{\pi(i)\pi(j)})_{i,j=1}^{N_{t}}$ for any permutation $\pi$ of $(1,2,\ldots,N_{t})$,
and that $\sum_{ij} X_{ij} = N_{t+1}$.
Note that the organisms could be hermaphrodite or unisexual,
as long as we assume that sex determination is independent of siblingship,
and so effectively make sex determination the first step of reproduction.


Each of these produces a \emph{random pedigree}, 
i.e.\ a directed graph with nodes indexed by $(t,k)$, for $1 \le k \le N_t$
and two types of edges, corresponding to maternal versus paternal relationships.
Each arrow represents one meiosis, i.e.\ the result of recombination and segration to produce a gamete.
If we can assume that genetic material does not affect mate choice,
then a model for a population can first choose a random pedigree,
then determine genetic relatedness by making the choices of recombination and segregation
independently in each meiosis.



%%%%%%% %%%%%%%%
\subsection*{Aside: The genome is not passive}

Of course, the genes an individual carries can have quite a strong influence
on their mate choice, number of offspring, and even on the outcome of recombination and segregation.
In particular, once we know the genomes of every individual, the Cannings model makes no sense,
since individuals are clearly not exchangeable.
This makes things much more complicated,
and we do our best to ignore it,
for instance, by imagining we are tracking only segments of the genome not under selection,
so that genetic variation in fitness only contributes to the distribution of offspring number.


%%%%%%% %%%%%%%%
\subsection{Diploid to haploid}

The diploid mpodels above involve a lot of bookkeeping,
that is done away with if we switch our focus from the individuals to the chromosomes.


\paragraph{The haploid Moran model:}
Continuous-time, fixed population size.
In continuous time, each individual, at rate 1, 
produces one offspring (through simple division)
that replaces a randomly chosen individual (possibly the parent).

\paragraph{The haploid Wright--Fisher model:}
Discrete-time, varying population size.
Suppose at time $t$ there is room for $N_t$ members of the population,
which we take as a given trajectory.
Each individual produces a Poisson number of offspring with large mean;
of the total pool of offspring, a uniformly chosen set of $N_{t+1}$ of these
are chosen to form the next generation.



